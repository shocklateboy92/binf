\documentclass[a4paper,12pt]{report}

\usepackage{setspace}
\onehalfspacing

\usepackage{anysize}
\marginsize{2cm}{2cm}{2cm}{2cm}

\usepackage{graphicx}
\usepackage{caption}
\usepackage{subcaption}
\usepackage{subfloat}
\usepackage[utf8]{inputenc}

\usepackage[dvips,pdftex]{hyperref}

\author{Lasath Fernando}
\title{Thesis A}
\date{15/10/14}

\begin{document}
\maketitle


\chapter{Introduction}
\section{Immune Response}
\section{Hidden Markov Models}
\section{Motivation}
\subsection{iHMMuneAlign}
Developed in 2007 by a team of researchers at UNSW, iHMMuneAlign attempts to simulate this process.
It represents the combination process of immunoglobulin heavy chain genes as a hidden markov model.
\subsection{Existing Solution Limitations}
\section{Aims}
\section{Propsed Solution}
Many things worth trying. In order, most promising to least
\end{document}
